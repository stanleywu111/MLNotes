\chapter{线性代数}
\section{向量}
\subsection{线性独立}
若:
\[\beta_1a_1+\cdots+\beta_ka_k=0\]
对于$ n $维向量$ \left \{ a_1,\,\cdots,a_k \right \} $存在不全为零的解$ \beta_1, \cdots,\beta_k $,则称$ \left \{ a_1,\,\cdots,a_k \right \} $线性相关。\newline
若:
\[\beta_1a_1+\cdots+\beta_ka_k=0\]
对于$ n $维向量$ \left \{ a_1,\,\cdots,a_k \right \} $的解为$ \beta_1=\cdots=\beta_k=0 $,则称$ \left \{ a_1,\,\cdots,a_k \right \} $线性独立。
\subsection{基}
一个线性独立的$ n $维向量$ \left \{ a_1,\,\cdots,a_n \right \} $称为基。$ n $维向量$ b $可以表示为基的线性组合:
\[ b=\alpha_1 a_1+\cdots+\alpha_n a_n \]
且系数$ \left \{ \alpha_1,\,\cdots,\alpha_n \right \} $唯一。
\subsection{Gram-Schmidt算法}
Gram-Schmidt算法用来判断$ n $维向量$ \left \{ a_1,\,\cdots,a_k \right \} $是否线性独立。

\section{矩阵}
\subsection{矩阵基本性质}
\begin{itemize}
	\item $ (AB)C = A(BC) $
	\item $ A(B+C) = AB +AC $
	\item $ k(AB)=(kA)B=A(kB) $
	\item $ (AB)^T=B^TA^T $
	\item $ AI=A; IA=A $
	\item $ AB=BA $一般情况下不成立
\end{itemize}
\subsection{矩阵的转置}
\begin{itemize}
	\item $ (A^T)^T=A$
	\item $ (A+B)^T=A^T=B^T $
	\item $ (kA)^T=kA^T $
	\item \fbox{$(AB)^T=B^TA^T$} 
\end{itemize}
若$ A^T=A $,则称$ A $是一个对称矩阵,若$ A^T=-A $,则称$ A $是一个反对称矩阵。
\subsection{矩阵的逆}
对方阵$ A $,若存在矩阵$ B $,满足$AB=BA=I$,则称$ A $是可逆的,称$ B $是$ B $的逆矩阵,记作$A^{-1}$。不可逆矩阵也成为奇异矩阵,可逆矩阵称为非奇异矩阵。
\begin{itemize}
	\item 若$ A $是可逆矩阵,则$ A^{-1} $也是可逆矩阵,且满足$ (A^{-1})^{-1}=A $
	\item $ (AB)^{-1}=B^{-1}A^{-1} $
	\item $ (A^T)^{-1}=(A^{-1})^T $
\end{itemize}


















