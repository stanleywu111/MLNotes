\chapter{线性代数}
\section{向量}
\subsection{线性独立}
若:
\[\beta_1a_1+\cdots+\beta_ka_k=0\]
对于$ n $维向量$ \left \{ a_1,\,\cdots,a_k \right \} $存在不全为零的解$ \beta_1, \cdots,\beta_k $,则称$ \left \{ a_1,\,\cdots,a_k \right \} $线性相关。\newline
若:
\[\beta_1a_1+\cdots+\beta_ka_k=0\]
对于$ n $维向量$ \left \{ a_1,\,\cdots,a_k \right \} $的解为$ \beta_1=\cdots=\beta_k=0 $,则称$ \left \{ a_1,\,\cdots,a_k \right \} $线性独立。
\subsection{基}
一个线性独立的$ n $维向量$ \left \{ a_1,\,\cdots,a_n \right \} $称为基。$ n $维向量$ b $可以表示为基的线性组合:
\[ b=\alpha_1 a_1+\cdots+\alpha_n a_n \]
且系数$ \left \{ \alpha_1,\,\cdots,\alpha_n \right \} $唯一。
\subsection{Gram-Schmidt算法}
Gram-Schmidt算法用来判断$ n $维向量$ \left \{ a_1,\,\cdots,a_k \right \} $是否线性独立。

\section{矩阵}
\subsection{矩阵基本性质}
\begin{itemize}
	\item $ (AB)C = A(BC) $
	\item $ A(B+C) = AB +AC $
	\item $ (AB)^T=B^TA^T $
	\item $ AI=A; IA=A $
	\item $ AB=BA $一般情况下不成立
\end{itemize}